\documentclass[svgnames,13pt]{beamer}
\usepackage[utf8]{inputenc}
\usepackage[T1]{fontenc}
\usepackage[greek,english]{babel}
\usepackage{textgreek}
\usefonttheme[onlymath]{serif}
\usetheme{Singapore}
\usecolortheme[RGB={50,104,20}]{structure}
%\setbeamertemplate{frametitle}[default][left]
\usepackage{verbatim}
\usepackage{eso-pic}
\usepackage{color}
\usepackage[autoplay,loop]{animate}
\usepackage{grffile}
\definecolor{emphasizecolor}{RGB}{255,69,0}
\graphicspath{{../common_media/}{./media/}}
\newcommand\fontvi{\fontsize{14}{15}\selectfont}
\newcommand {\framedgraphic}[3] {
	\begin{frame}{#1}
		\begin{center}
			\includegraphics[width=\textwidth,height=#3\textheight,keepaspectratio]{#2}
		\end{center}
	\end{frame}
}
\makeatletter
\let\beamer@writeslidentry@miniframeson=\beamer@writeslidentry
\def\beamer@writeslidentry@miniframesoff{%
  \expandafter\beamer@ifempty\expandafter{\beamer@framestartpage}{}% does not happen normally
  {%else
    % removed \addtocontents commands
    \clearpage\beamer@notesactions%
  }
}

% change default resolution of PNG imags to 600
% (doesn't seem to do anything)
%\setkeys{Gin}{resolution=600}

% math voodo... just want bold math to work
\usepackage{amsmath}
\usepackage{amsthm}
\usepackage{eucal}
\usepackage{amssymb}
\usepackage{mathrsfs}

% footnotes too big
\let\oldfootnotesize\footnotesize
\renewcommand*{\footnotesize}{\oldfootnotesize\TINY}

\newcommand*{\miniframeson}{
    \let\beamer@writeslidentry=\beamer@writeslidentry@miniframeson
}
\newcommand*{\miniframesoff}{
    \let\beamer@writeslidentry=\beamer@writeslidentry@miniframesoff
}
\makeatother
\makeatletter
\DeclareRobustCommand{\emph}[1]{\textbf{{\color{emphasizecolor} #1}}}
\makeatother

\newcommand{\argmin}{\operatornamewithlimits{argmin}}
\newcommand{\argmax}{\operatornamewithlimits{argmax}}

\begin{document}


\title   {Review of generalized linear point-process models}
\author  {Neurostats club 2015}
\subtitle{}
\date    {\today}
\begin{frame} \titlepage \end{frame}

%#############################################################################
%#############################################################################
\section{Introduction}


%=============================================================================
\subsection{}

%_____________________________________________________________________________
\begin{frame}{}
\begin{center}
	\large This will be quick\\
	No formal derivation\\
	No proofs
\end{center}
\end{frame} 

%_____________________________________________________________________________
\begin{frame}{Truccolo et al. 2005}
\begin{overprint}
	\onslide<1>
	\begin{itemize}
		\item Derivation of point process GLM
		\item Intrinsic and ensemble history models
		\item Maximum likelihood estimation
		\item Non-GLM conditional intensity model
		\item KS test and time rescaling
		\item Residual analysis
		\item Model selection
		\item Decoding
		\end{itemize}
	\onslide<2>
	\begin{itemize}
		\item Derivation of point process GLM
		\item \emph{Intrinsic and ensemble history models}
		\item \emph{Maximum likelihood estimation}
		\item Non-GLM conditional intensity model
		\item KS test and time rescaling
		\item Residual analysis
		\item Model selection
		\item Decoding
		\end{itemize}
	\onslide<3,4>
	\begin{itemize}
		\item Derivation of point process GLM
		\item \emph{Intrinsic and ensemble history models}
		\item \emph{Maximum likelihood estimation}
		\item Non-GLM conditional intensity model
		\item KS test and time rescaling
		\item Residual analysis
		\item Model selection
		\item Decoding
		\item \emph{Take NEUR2110 with Wilson Truccolo in the spring!}
		\end{itemize}
		\end{overprint}
\begin{overprint}
	\onslide<4>
	If there is time:
	\begin{itemize}
		\item Closed form approximation, double crossvalidation, regularization, and regularization paths
	\end{itemize}
\end{overprint}
\end{frame} 

%_____________________________________________________________________________
\begin{frame}{Linear model / multiple linear regression}
\[y = \beta_o + \beta_1 \cdot x_1 + \beta_2 \cdot x_2 + ... = \beta_o + \sum_{i=1}^N x_i \beta_i \ = XB + \beta_o\]
\begin{itemize}
	\item \emph{$y$}: response/dependent variable being modeled (when multivariate, often a column vector by convention)
	\item \emph{$X$}: ``design matrix'' matrix of observations. Conventionally, each row is a realization and each column is a feature/variable
	\item \emph{$B$}: Coefficient or parameter vector
	\item \emph{$\beta_o$}: Constant term.
\end{itemize}
\end{frame} 

%_____________________________________________________________________________
\begin{frame}{Nonlinear features OK}
\begin{itemize}
	\item Features $(x_1,x_2,...)$ can be any function of recorded data. E.g. the following model is common for phase/direction tuning (used in eqn. 10 in Truccolo et al. 2005)
	\end{itemize}
\[y = \beta_o + A\cdot cos(\varphi-\varphi_o)\]
\vspace{-2em}
\begin{itemize}
	\item <2->\emph{$A$}: amplitude parameter
	\item <3->\emph{$\varphi$}: the observed phase or direction
	\item <4->\emph{$\varphi_o$}: preferred phase parameter
	\item <5->Can be written in a form that is linear in parameters
	\end{itemize}
\begin{overprint}
	\onslide<5>
	\[y = \beta_o + A cos(\varphi_o) cos(\varphi) +  A sin(\varphi_o) sin(\varphi)\]
	\onslide<6>
	\[y = \beta_o + \beta_1 cos(\varphi) + \beta_2 sin(\varphi)\]
\end{overprint}
\end{frame} 


%#############################################################################
%#############################################################################
\section{GLM point process for spike trains}


%=============================================================================
\subsection{ }

%_____________________________________________________________________________
\begin{frame}{Point process model:}
\begin{itemize}
	\item Truccolo et al. 2005:
	\begin{itemize}
		\item <1->``neural spike trains form a sequence of discrete events or point process time series''
		\item <2->``standard linear or nonlinear regression methods are designed for analysis of continuous-valued data and not point process observations''
		\end{itemize}
	\item <3->Smoothing or binning can alter the structure
	\item <4->GLM Point-process models directly model spike trains without these drawbacks
\end{itemize}
\end{frame} 

%_____________________________________________________________________________
\begin{frame}{GLM point process models}
\begin{itemize}
	\item <1->Consider a homogeneous Poisson process with rate $\lambda$. 
	\begin{itemize}
		\item expected \# observations in time window $\Delta$ is $\lambda\Delta$
		\end{itemize}
	\item <2->For inhomogeneous Poisson process rate varies time $\lambda(t)$
	\begin{itemize}
		\item \# observations from $t$ to $t+\Delta$ is $\int_t^{t+\Delta} \lambda(t) dt$
		\end{itemize}
	\item <3->$\lambda(t)$ is called the "intensity function"
	\item <4->In a point process model we predict \emph{conditional intensity} based on measured covariates $\lambda(t|X(t))$
	\item <5->The Poisson GLM point process framework models conditional intensity functions of the form
	\end{itemize}
\begin{overprint}
	\onslide<5>
	\[\lambda(t|X(t)) = \exp \left\{ \mu + X B \right\}\]
\end{overprint}
\end{frame} 

%_____________________________________________________________________________
\begin{frame}{Fitting GLM point-process models   }
\begin{itemize}
	\item <1->In practice spiking time series are discretized at some resolution $\Delta$
	\begin{itemize}
		\item <2->Predict sequence of spike counts $y_t=(y_1,y_2,...)$
		\item <3->With a discrete approximation of the conditional intensity $\lambda_t=(\lambda_1,\lambda_2,...)$
		\item <4->Based on discretely sampled covariates $X_t=(x_1,x_2,...)$
		\end{itemize}
	\item <5->Typically $\Delta$ chosen such that $y_t\in\{0,1\}$, (larger bins can be used, making it a count process)
	\item <6->Model estimated by finding parameters $B$ that maximize the likelihood of the observed spikes $y$ and covariates $X$
	\begin{itemize}
		\item <7->Can be fit with iteratively reweighted least squares 
		\item <8->MATLAB glmfit \\ \href{http://www.mathworks.com/help/stats/glmfit.html}{\beamergotobutton{mathworks.com/help/stats/glmfit.htmlt}}
		\item <9->Python scikits.statsmodels.GLM \href{http://statsmodels.sourceforge.net/stable/glm.html}{\beamergotobutton{statsmodels.sourceforge.net/stable/glm.html}} 
		\item <10->I typically use gradient descent
	\end{itemize}
\end{itemize}
\end{frame} 

%_____________________________________________________________________________
\begin{frame}{GLM point process model for single unit spiking with ensemble history}
\begin{overprint}
	\onslide<1>
	\[\begin{aligned}\log[\lambda(t|X(t))\Delta] &= \mu \\
	&\phantom{{}+ \text{\emph{intrinsic history}}} \\
	&\phantom{{}+ \text{\emph{ensemble history}}}\\
	&\phantom{{}+ \text{\emph{extrinsic covariates}}}\end{aligned}\]
	\onslide<2>
	\[\begin{aligned}\log[\lambda(t|X(t))\Delta] &= \mu \\
	&{}+ \text{\emph{intrinsic history}} \\
	&\phantom{{}+ \text{\emph{ensemble history}}}\\
	&\phantom{{}+ \text{\emph{extrinsic covariates}}}\end{aligned}\]
	\onslide<3>
	\[\begin{aligned}\log[\lambda(t|X(t))\Delta] &= \mu \\
	&{}+ \text{\emph{intrinsic history}} \\
	&{}+ \text{\emph{ensemble history}}\\
	&\phantom{{}+ \text{\emph{extrinsic covariates}}}\end{aligned}\]
	\onslide<4>
	\[\begin{aligned}\log[\lambda(t|X(t))\Delta] &= \mu \\
	&{}+ \text{\emph{intrinsic history}} \\
	&{}+ \text{\emph{ensemble history}}\\
	&{}+ \text{\emph{extrinsic covariates}}\end{aligned}\]
\end{overprint}
\end{frame} 

%_____________________________________________________________________________
\begin{frame}{Intrinsic history filter}
\begin{itemize}
	\item <1->Theoretically, model the whole history \[\lambda_t(t|\theta,y_{t-1},y_{t-2},...,y_1) \propto \exp\left\{ \sum_{\tau=1}^t \gamma_\tau y_{t-\tau} \right\}\]
	\item <2->In practice: use finite history duration with \emph{basis functions} (a form of regularization: enforce smoothness and reduce parameters) \[\lambda_t(t|\theta,y_{t-\tau:t-1}) \propto \exp\left\{ B \cdot y_{t-\tau:t-1} \right\}\]
	\end{itemize}
\begin{overprint}
	\onslide<2>
	\begin{figure}\centering\includegraphics[width=1.00\textwidth,height=0.30\textheight,keepaspectratio]{./20150301 inhouse/20150301 old inhouse2015 EMPTY?/media/glm_animation/intrinsic_basis.pdf}\\\end{figure}
\end{overprint}
\end{frame} 

%_____________________________________________________________________________
\begin{frame}{Ensemble history filter}
\begin{itemize}
	\item <1->Ensemble history filters are treated similarly to intrinsic history
	\item <2->Typically use fewer basis functions than for the intrinsic history, otherwise the number of parameters becomes prohibitive
	\item <3->If using regularization, typically each neuron's parameters are penalized as a group (sparse connectivity prior)
	\item <4->More on intrinsic history and network connectivity next week
\end{itemize}
\end{frame} 

%_____________________________________________________________________________
\begin{frame}{Extrinsic covariates: kinematics}
\begin{itemize}
	\item <1->Truccolo et al. 2005 explore a 2D velocity tuning model based on the $x$ and $y$ components of hand velocity. 
	\item <2->Hatsopoulos et al. 2007 use normalized, extended velocity trajectories "pathlets"
\end{itemize}
\end{frame} 

%_____________________________________________________________________________
\begin{frame}{Maximum likelihood approach to model fitting}
\begin{itemize}
	\item <1->Let $y_t$ be an inhomogeneous Poisson with time varying rate $\lambda_t$
	\item <2->The probability of observing $k$ spikes in a time interval $\Delta$ is Poisson distributed\[\Pr(y_t=k) \approx (\Delta \lambda_t)^{y_t} \frac{ e^{-\Delta \lambda_t}}{y_t!}\]
	\item <3->Assuming conditional independence, the probability of observing an entire sequence $y_t$ is \[\Pr(y|\lambda) = \prod_{t=1}^T (\Delta\lambda_t)^{y_t} e^{-\Delta\lambda_t} / {y_t}! \]
\end{itemize}
\end{frame} 

%_____________________________________________________________________________
\begin{frame}{Minimize the negative log-likelihood}
\begin{itemize}
	\item <1->Fit the model by finding the parameters $\mu$, $B$ that \emph{maximize the likelihood} $\mathcal{L}(\mu,B|y) = \Pr(y|\mu,B)$ of the observations\footnote{Note: writing $\lambda_t$ here instead of $\Delta\lambda_t$, i.e. let $\Delta=1$. In this case, parameters $\mu$ and $B$ will take units of $\Delta$} \[\Pr(y|\mu,B) = \prod_{t=1}^T \lambda_t^{y_t} e^{-\lambda_t} / {y_t}! \]\[\lambda_t=\exp(\mu + X_t B)\]
	\item <2->In practice, \emph{minimize the negative log-likelihood}, which, if $\Delta$ is small $s.t.$ $y_t$ is always 0 or 1 \[ -\ln\mathcal{L}(\mu,B|y) =  \sum_{t=1}^T[ \lambda_t - y_t \ln(\lambda_t) ]\]
\end{itemize}
\end{frame} 

%_____________________________________________________________________________
\begin{frame}{Gradient of the negative log-likelihood}
\begin{itemize}
	\item <1->Let $f(\mu,B)$ be the negative log likelihood \[f(\mu,B) = -\ln\mathcal{L}(\mu,B|y) =  \sum_{t=1}^T[ \lambda_t - y_t \ln(\lambda_t) ]\]
	\item <2->Substitute our model $\lambda_t=e^{\mu + X_t B}$ \[f(\mu,B) =  \sum_{t=1}^T[ e^{\mu + X_t B} - y_t (\mu + X_t B) ]\]
	\item <3->The partial derivatives, w.r.t $\mu$ and $(\beta_1,\beta_2,...)=B$ are:\[\begin{aligned}\frac{\partial f}{\partial	\mu}=&\sum_{t=1}^T[e^{\mu+X_t B}-y_t]\\\frac{\partial f}{\partial\beta_i}=&\sum_{t=1}^T[X_{t,i} e^{\mu+X_t B}-y_t X_{t,i}]\end{aligned}\]
\end{itemize}
\end{frame} 


%=============================================================================
\subsection{Regularization}

%_____________________________________________________________________________
\begin{frame}{Regularized GLM }
\begin{itemize}
	\item <1->Cross-validation is necessary to assess over-fitting: 
	\begin{itemize}
		\item data should be separated into training and testing sets
		\end{itemize}
	\item <2->For larger number of parameters, the model \emph{will} overfit:
	\begin{itemize}
		\item regularization is necessary. 
		\end{itemize}
	\item <3->Regularization can be incorporated by adding a penalty term to the negative log-likelihood function \[\argmin_{\mu,B}\left\{ \textrm{\emph{Penalty}}(B) - \ln\mathcal{L}(\mu,B|y) \right\}\]
	\item <4->Conjugate gradient solvers are useful here
\end{itemize}
\end{frame} 

%_____________________________________________________________________________
\begin{frame}{Regularized GLM: L1 and L2}
\begin{itemize}
	\item <1->L2 penalty: Parameters penalized by their squared magnitudes\[  \alpha \sum_{i=1}^N \beta_i^2 \]
	\begin{itemize}
		\item <2->Equivalent to a Gaussian prior on parameters
		\item <3->Can be solved with gradient descent
		\end{itemize}
	\item <4->L1 penalty: Parameters penalized by their absolute magnitude\[ \alpha \sum_{i=1}^N |\beta_i| \]
	\begin{itemize}
		\item <5->Promotes $\beta_i=0$, useful for finding sparse solutions
		\item <6->Discontinuous gradient at $\beta_i=0$ percludes gradient descent. 
		\item <7->Can use coordinate descent (although we ran into convergence issues?)
	\end{itemize}
\end{itemize}
\end{frame} 

%_____________________________________________________________________________
\begin{frame}{Regularized GLM: L1 approximation and L0}
\begin{itemize}
	\item <1->~$\sqrt{x^2+\epsilon}:$ Smooth approximation of L1 penalty that is suitable for gradient descent\footnote{Called "Charbonnier penalty" in the computer vision literature (Charbonnier et al. 1994)}\[ \alpha \sum_{i=1}^N \sqrt{\beta_i^2+\varepsilon} \]
	\begin{itemize}
		\item <2->$\epsilon$ is chosen to be small, but strictly positive
		\end{itemize}
	\item <3->L0 penalty: Constant penalty if a parameter is nonzero\[ \alpha \sum_{i=1}^N \delta(\beta_i\ne 0) \]
	\begin{itemize}
		\item <4->Computationally infeasible
		\item <5->Greedy algorithms are a good (the best polynomial time?) approximation
	\end{itemize}
\end{itemize}
\end{frame} 

%_____________________________________________________________________________
\begin{frame}{Regularized GLM: Group lasso}
\begin{itemize}
	\item <1->Concept: penalize \emph{groups} of parameters with the L1 norm\[ \alpha \sum_{i=1}^N \sqrt{ \sum_{j=1}^K \beta_{i,j}^2 } \]
	\begin{itemize}
		\item <2->Useful for penalizing ensemble filters as a group for each neuron
		\item <3->Derivative is undefined at $\sqrt{ \sum_{j=1}^K \beta_{i,j}^2 }=0$
		\item <4->Roth \& Fischer 2008\footnote{Roth, Volker, and Bernd Fischer. "The group-lasso for generalized linear models: uniqueness of solutions and efficient algorithms." Proceedings of the 25th international conference on Machine learning. ACM, 2008.} discuss an efficient fitting procedure
	\end{itemize}
\end{itemize}
\end{frame} 

%_____________________________________________________________________________
\begin{frame}{Two-layer crossvalidation}
\begin{itemize}
	\item <1->The strength of regularization $\alpha$ is another free parameter
	\begin{itemize}
		\item Can be fixed in advance or...
		\item <2->Two-level crossvalidation can be used to estimate $\alpha$ from the data
		\end{itemize}
	\item <3->K-fold two-layer crossvalidation results in K${}^2$ parameter fitting steps
	\begin{itemize}
		\item This can get slow on large datasets
	\end{itemize}
\end{itemize}
\end{frame} 

%_____________________________________________________________________________
\begin{frame}{Regularization paths}
\begin{itemize}
	\item Regularization paths:
	\begin{itemize}
		\item When evaluating a range of regularization parameters e.g. $\alpha=(0,0.1,1,20)$,
		\item Carry over the model weights $\mu$ and $\beta$. 
		\item That is, for example, start the parameter tuning for $\alpha=1$ with the $\mu,B$ returned from $\alpha=0.1$
	\end{itemize}
\end{itemize}
\end{frame} 

%_____________________________________________________________________________
\begin{frame}{Initialization with closed form solution}
\begin{itemize}
	\item <1->Fitting many models using a large amount of data can take a very long time
	\item <2->Using an approximation to estimate a good starting location can give some speedup
	\item <2->Ramirez, A. D., \& Paninski, L. (2014). Fast inference in generalized linear models via expected log-likelihoods. Journal of computational neuroscience, 36(2), 215-234.
\end{itemize}
\end{frame} 

%_____________________________________________________________________________
\begin{frame}{}
\begin{center}
	Truccolo et al. 2005 
\end{center}
\end{frame} 


%#############################################################################
%#############################################################################
\section{Truccolo et al. 2005 figures and equations}


%=============================================================================
\subsection{ Eqn}

%_____________________________________________________________________________
\begin{frame}{}
\begin{figure}\centering\includegraphics[scale=0.14]{./media/truccolo2005_equations_figures/truccolo2005eq1.png}\\\end{figure}
\end{frame} 

%_____________________________________________________________________________
\begin{frame}{}
\begin{figure}\centering\includegraphics[scale=0.14]{./media/truccolo2005_equations_figures/truccolo2005eq2.png}\\\end{figure}
\end{frame} 

%_____________________________________________________________________________
\begin{frame}{}
\begin{figure}\centering\includegraphics[scale=0.14]{./media/truccolo2005_equations_figures/truccolo2005eq3.png}\\\end{figure}
\end{frame} 

%_____________________________________________________________________________
\begin{frame}{}
\begin{figure}\centering\includegraphics[scale=0.14]{./media/truccolo2005_equations_figures/truccolo2005eq4.png}\\\end{figure}
\end{frame} 

%_____________________________________________________________________________
\begin{frame}{}
\begin{figure}\centering\includegraphics[scale=0.14]{./media/truccolo2005_equations_figures/truccolo2005eq5.png}\\\end{figure}
\end{frame} 

%_____________________________________________________________________________
\begin{frame}{}
\begin{figure}\centering\includegraphics[scale=0.14]{./media/truccolo2005_equations_figures/truccolo2005eq6.png}\\\end{figure}
\end{frame} 

%_____________________________________________________________________________
\begin{frame}{}
\begin{figure}\centering\includegraphics[scale=0.14]{./media/truccolo2005_equations_figures/truccolo2005eq7.png}\\\end{figure}
\end{frame} 

%_____________________________________________________________________________
\begin{frame}{}
\begin{figure}\centering\includegraphics[scale=0.14]{./media/truccolo2005_equations_figures/truccolo2005eq8.png}\\\end{figure}
\end{frame} 

%_____________________________________________________________________________
\begin{frame}{}
\begin{figure}\centering\includegraphics[scale=0.14]{./media/truccolo2005_equations_figures/truccolo2005eq9.png}\\\end{figure}
\end{frame} 

%_____________________________________________________________________________
\begin{frame}{}
\begin{figure}\centering\includegraphics[scale=0.14]{./media/truccolo2005_equations_figures/truccolo2005eq10.png}\\\end{figure}
\end{frame} 

%_____________________________________________________________________________
\begin{frame}{}
\begin{figure}\centering\includegraphics[scale=0.14]{./media/truccolo2005_equations_figures/truccolo2005eq11.png}\\\end{figure}
\end{frame} 

%_____________________________________________________________________________
\begin{frame}{}
\begin{figure}\centering\includegraphics[scale=0.14]{./media/truccolo2005_equations_figures/truccolo2005eq12.png}\\\end{figure}
\end{frame} 

%_____________________________________________________________________________
\begin{frame}{}
\begin{figure}\centering\includegraphics[scale=0.14]{./media/truccolo2005_equations_figures/truccolo2005eq13.png}\\\end{figure}
\end{frame} 

%_____________________________________________________________________________
\begin{frame}{}
\begin{figure}\centering\includegraphics[scale=0.14]{./media/truccolo2005_equations_figures/truccolo2005eq14.png}\\\end{figure}
\end{frame} 

%_____________________________________________________________________________
\begin{frame}{}
\begin{figure}\centering\includegraphics[scale=0.14]{./media/truccolo2005_equations_figures/truccolo2005eq15.png}\\\end{figure}
\end{frame} 

%_____________________________________________________________________________
\begin{frame}{}
\begin{figure}\centering\includegraphics[scale=0.14]{./media/truccolo2005_equations_figures/truccolo2005eq16.png}\\\end{figure}
\end{frame} 

%_____________________________________________________________________________
\begin{frame}{}
\begin{figure}\centering\includegraphics[scale=0.14]{./media/truccolo2005_equations_figures/truccolo2005eq17.png}\\\end{figure}
\end{frame} 

%_____________________________________________________________________________
\begin{frame}{}
\begin{figure}\centering\includegraphics[scale=0.14]{./media/truccolo2005_equations_figures/truccolo2005eq18.png}\\\end{figure}
\end{frame} 

%_____________________________________________________________________________
\begin{frame}{}
\begin{figure}\centering\includegraphics[scale=0.14]{./media/truccolo2005_equations_figures/truccolo2005eq19.png}\\\end{figure}
\end{frame} 

%_____________________________________________________________________________
\begin{frame}{}
\begin{figure}\centering\includegraphics[scale=0.14]{./media/truccolo2005_equations_figures/truccolo2005eq20.png}\\\end{figure}
\end{frame} 

%_____________________________________________________________________________
\begin{frame}{}
\begin{figure}\centering\includegraphics[scale=0.14]{./media/truccolo2005_equations_figures/truccolo2005eq21.png}\\\end{figure}
\end{frame} 

%_____________________________________________________________________________
\begin{frame}{}
\begin{figure}\centering\includegraphics[scale=0.14]{./media/truccolo2005_equations_figures/truccolo2005eq22.png}\\\end{figure}
\end{frame} 


%=============================================================================
\subsection{ FIG}

%_____________________________________________________________________________
\begin{frame}{}
\begin{figure}\centering\includegraphics[width=1.00\textwidth,height=0.85\textheight,keepaspectratio]{./media/truccolo2005_equations_figures/truccolo2005f1.png}\\\end{figure}
\end{frame} 

%_____________________________________________________________________________
\begin{frame}{}
\begin{figure}\centering\includegraphics[width=1.00\textwidth,height=0.85\textheight,keepaspectratio]{./media/truccolo2005_equations_figures/truccolo2005f2.png}\\\end{figure}
\end{frame} 

%_____________________________________________________________________________
\begin{frame}{}
\begin{figure}\centering\includegraphics[width=1.00\textwidth,height=0.85\textheight,keepaspectratio]{./media/truccolo2005_equations_figures/truccolo2005f3.png}\\\end{figure}
\end{frame} 

%_____________________________________________________________________________
\begin{frame}{}
\begin{figure}\centering\includegraphics[width=1.00\textwidth,height=0.85\textheight,keepaspectratio]{./media/truccolo2005_equations_figures/truccolo2005f4.png}\\\end{figure}
\end{frame} 

%_____________________________________________________________________________
\begin{frame}{}
\begin{figure}\centering\includegraphics[width=1.00\textwidth,height=0.85\textheight,keepaspectratio]{./media/truccolo2005_equations_figures/truccolo2005f5.png}\\\end{figure}
\end{frame} 

%_____________________________________________________________________________
\begin{frame}{}
\begin{figure}\centering\includegraphics[width=1.00\textwidth,height=0.85\textheight,keepaspectratio]{./media/truccolo2005_equations_figures/truccolo2005f6.png}\\\end{figure}
\end{frame} 

%_____________________________________________________________________________
\begin{frame}{}
\begin{figure}\centering\includegraphics[width=1.00\textwidth,height=0.85\textheight,keepaspectratio]{./media/truccolo2005_equations_figures/truccolo2005f7.png}\\\end{figure}
\end{frame} 

%_____________________________________________________________________________
\begin{frame}{}
\begin{figure}\centering\includegraphics[width=1.00\textwidth,height=0.85\textheight,keepaspectratio]{./media/truccolo2005_equations_figures/truccolo2005f8.png}\\\end{figure}
\end{frame} 

%_____________________________________________________________________________
\begin{frame}{}
\begin{figure}\centering\includegraphics[width=1.00\textwidth,height=0.85\textheight,keepaspectratio]{./media/truccolo2005_equations_figures/truccolo2005f9.png}\\\end{figure}
\end{frame} 

%_____________________________________________________________________________
\begin{frame}{}
\begin{figure}\centering\includegraphics[width=1.00\textwidth,height=0.85\textheight,keepaspectratio]{./media/truccolo2005_equations_figures/truccolo2005f10.png}\\\end{figure}
\end{frame} 


%=============================================================================
\subsection{ APPENDIX}

%_____________________________________________________________________________
\begin{frame}{}
\begin{figure}\centering\includegraphics[scale=0.14]{./media/truccolo2005_equations_figures/truccolo2005eqA1.png}\\\end{figure}
\end{frame} 

%_____________________________________________________________________________
\begin{frame}{}
\begin{figure}\centering\includegraphics[scale=0.14]{./media/truccolo2005_equations_figures/truccolo2005eqA2.png}\\\end{figure}
\end{frame} 

%_____________________________________________________________________________
\begin{frame}{}
\begin{figure}\centering\includegraphics[scale=0.14]{./media/truccolo2005_equations_figures/truccolo2005eqA3.png}\\\end{figure}
\end{frame} 

%_____________________________________________________________________________
\begin{frame}{}
\begin{figure}\centering\includegraphics[scale=0.14]{./media/truccolo2005_equations_figures/truccolo2005eqA4.png}\\\end{figure}
\end{frame} 

%_____________________________________________________________________________
\begin{frame}{}
\begin{figure}\centering\includegraphics[scale=0.14]{./media/truccolo2005_equations_figures/truccolo2005eqA5.png}\\\end{figure}
\end{frame} 

%_____________________________________________________________________________
\begin{frame}{}
\begin{figure}\centering\includegraphics[scale=0.14]{./media/truccolo2005_equations_figures/truccolo2005eqA6.png}\\\end{figure}
\end{frame} 

%_____________________________________________________________________________
\begin{frame}{}
\begin{figure}\centering\includegraphics[scale=0.14]{./media/truccolo2005_equations_figures/truccolo2005eqA7.png}\\\end{figure}
\end{frame} 

%_____________________________________________________________________________
\begin{frame}{}
\begin{figure}\centering\includegraphics[scale=0.14]{./media/truccolo2005_equations_figures/truccolo2005eqA8.png}\\\end{figure}
\end{frame} 

%_____________________________________________________________________________
\begin{frame}{}
\begin{figure}\centering\includegraphics[scale=0.14]{./media/truccolo2005_equations_figures/truccolo2005eqA9.png}\\\end{figure}
\end{frame} 

%_____________________________________________________________________________
\begin{frame}{}
\begin{figure}\centering\includegraphics[scale=0.14]{./media/truccolo2005_equations_figures/truccolo2005eqA10.png}\\\end{figure}
\end{frame} 

%_____________________________________________________________________________
\begin{frame}{}
END
\end{frame} 


%#############################################################################
%#############################################################################
%#############################################################################
\appendix\newcounter{finalframe}
\setcounter{finalframe}{\value{framenumber}}
\setcounter{framenumber}{\value{finalframe}}

%_____________________________________________________________________________
\begin{frame}{Generalized linear model: nonlinearity on the response variable}
\[F(y) = XB\]
\end{frame} 

%_____________________________________________________________________________
\begin{frame}{Regularized GLM: Elastic net}
\begin{itemize}
	\item <1->Concept: dynamic trade-off between L1 and L2
	\begin{itemize}
		\item <2-> Code crashed pretty hard. 
	\end{itemize}
\end{itemize}
\end{frame} 



\end{document}
